\documentclass[12pt]{ctexart}
% 五号 -- 10.5pt, 小四号 -- 12pt, 四号 -- 14pt
% 一般12pt即可

\usepackage{graphicx}
% 引入graphicx宏包以实现插入图片的功能

\usepackage{geometry}
\geometry{left=2cm, right=2.5cm, top=0.5cm, bottom=1cm}
% 引入geometry宏包并调整上下左右的页边距

\pagestyle{empty}
% 去除页眉页脚

\CTEXsetup[format={\Large\bfseries},
			beforeskip = 0.2cm
			]{section}
% format参数的内容是让标题左对齐而不是居中
% beforeskip参数的内容是调整下一节标题与上一节
% 正文的间距

\begin{document}
%----------第一部分: 基础信息-------------
	% 这个minipage写的是姓名, 手机, 邮箱, 微信等信息
	\noindent
	% 这里使用\noindent取消缩进, 是因为可以把minipage
	% 当成一段新的文字开头, 否则表格前面会有2em的空白
	\begin{minipage}{0.5\linewidth}
		{\zihao{2} \kaishu 黄非凡}
		\vskip 0.2cm
			\begin{tabular}{rl}
				\textbf{手机号码:} & 183-1913-9470 \\ 
				\textbf{邮箱地址:} & 17ffhuang@stu.edu.cn \\ 
				\textbf{微信号:} & wxid\_g163m3w8jcxe12 \\ 
			\end{tabular}
	\end{minipage}
	% 这个minipage装的是你的头像
	% 注意用\hfill实现右对齐的效果
	% 注意2个minipage之间不要有空行
	\begin{minipage}{0.5\linewidth}
		\hfill \raisebox{30pt}{\includegraphics[scale = 0.13]{ff.jpg}}
		% \raisebox的作用是上下微调你的头像的位置
	\end{minipage}

%-----------第二部分: 教育经历----------------
	\vspace{-0.6cm}
	\section*{教育经历}
	\vspace{-0.2cm}
	\hrule
	\vskip 0.2cm
	\noindent \textbf{汕头大学}\quad 理学院 
	\hfill 2017.09 -- 2021.06 \\
	\textbf{数学与应用数学专业} \quad \textbf{本科} \\
	GPA: 3.59 \quad 专业排名: 14
	
%-----------第三部分: 主修课程----------------
	\section*{主修课程}
	\vspace{-0.2cm}
	\hrule
	\vskip 0.2cm
	\noindent
	% 这里使用noindent取消整个表格前面的缩进
	\begin{tabular}{c|l}
		\textbf{分析方向} &
		微积分\quad
		数学分析I\&II(93和96)\quad 
		复变函数\quad 
		实变函数(91)\quad 
		泛函分析等 \\
		\textbf{代数方向} & 
		线性代数与解析几何I\&II \quad
		抽象代数\quad 
		初等数论等\quad \\
		\textbf{应用方向} & 
		概率论(96)\quad 
		数学建模(94)\quad
		图像处理(91)\quad
		机器人与智能计算(95)等				
	\end{tabular}
	\vskip 0.5cm
	
%-----------第四部分: 求职意愿----------------	
	\section*{\heiti 求职意愿}
	\vspace{-0.2cm}
	\hrule
	\vskip 0.2cm
	算法工程师
	
%-----------第五部分: 参赛与获奖经历----------------
	\section*{获奖经历}
	\vspace{-0.2cm}
	\hrule
	\vskip 0.2cm
	\begin{itemize}
		\item \textbf{2018 -- 2019学年本科生奖学金}
		\hfill 2019.12 \\
		\textbf{二等学业优秀奖学金}\qquad 本学年GPA排名: 4
		
		\vspace{-0.2cm}
		
		\item \textbf{第七届"泰迪杯"数据挖掘挑战赛}
		\hfill 2019.07 \\
		全国赛区\textbf{二等奖}\qquad 
		广东赛区\textbf{一等奖}\\
		作品名称: 直肠癌淋巴结转移的智能诊断
	\end{itemize}
	\vspace{-0.5cm}
	
%-----------第六部分: 实践经历----------------
	\section*{实践经历}
	\vspace{-0.2cm}
	\hrule
	\vskip 0.2cm
	\begin{itemize}
		\item \textbf{智创未来 --- 人工智能进小学}
			\hfill
			2019.06 -- 2019.07 \\
			公益课程志愿者\\
			在汕头市陈厝合小学为小学生讲授人工智能
	\end{itemize}
	\vspace{-0.5cm}

%-----------第七部分: 技术能力----------------
	\section*{技术能力}
	\vspace{-0.2cm}
	\hrule
	\vskip 0.2cm
	\begin{itemize}
		\item 能够熟练使用
		\includegraphics[height=0.36cm, width=0.36cm]
		{word.png}
		Word,\quad
		\includegraphics[height=0.36cm, width=0.36cm]
		{ppt.png}
		PowerPoint,\quad
		\includegraphics[height=0.36cm, width=0.36cm]
		{excel.png}
		Excel
		等软件进行日常办公\\
		\phantom{哈哈}代表作: 本科毕业论文《许瓦兹引理》
		
		\vspace{-0.2cm}
		
		\item 能够使用
		\includegraphics[height=0.36cm, width=0.36cm]
		{matlab.png}
		Matlab实现算法和计算数学模型\\
		\phantom{空白}代表作: 图像加密课程论文
		《基于双Logistic混沌映射的数字图像加密算法实现》
		
		\vspace{-0.2cm}
		
		\item 普通话(母语)\qquad 
		粤语(流利)\qquad  
		英语(CET4: 565)
	\end{itemize}
\end{document}